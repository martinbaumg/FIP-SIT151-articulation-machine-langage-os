%*****************************************************************************************************            
%                               INSTITUT MINES-TELECOM, IMT ATlantique
%                         Département Mathematical & Electrical Engineering           
%                                     Technopôle de Brest-Iroise                   
%                                  CS 83818 - 29238 BREST Cedex 3                       
%            
%                                    **  Tous droits réservés **
%                                      
%
% NOM DU PROJET : TP LaTeX 006 / Modèle de rapport de recherche (validé par la DRI), de support de cours
% et de TP au format LaTeX avec les éléments de la charte graphique de la communication d'IMT Atlantique
%
% LIVRABLE : [ ] Memo, [ ] Résumé, [ ] Article, [ ] Code, [X] Rapport , [ ] Thèse, [ ] Cours/TP
%
% AUTEUR(s) : Thierry LE GALL (Dépt. MEE), Guillaume ANSEL (SC), Thomas GUILMENT (Dépt. SC)
%
% HISTORIQUE :
%
% Date          Nom(s)          Version  Description
% ------------  --------------- -------  ----------------------------------------------------------
% 12-mar-2015   T. LE GALL (SC) 0.0      création du document 
% 26-mar-2016   T. LE GALL (SC) 0.1      mise à jour : essai du package UTF8
% 26-sep-2016   G. ANSEL        0.2      mise à jour : ajout exemple utilisation TBannexe
%                                        ajout correction titres en majuscule dans sommaire
%                                        (cf BUGFIX_TITRES_SOMMAIRE dans imta.sty)
%                                        changement polices utilisées
% 09-jan 2017   T. LE GALL (SC) 0.3      mise à jour : évolution du modèle de TB vers IMT Atlantique
% 03-fev 2017   G. ANSEL        0.4      mise à jour : modifications pour adaptation à version 1.5 du package imta
%                                        ajout méta-données PDF
% 14-fev-2017   T. GUILMENT     0.5      mise à jour : ajout package fancyhdr pour entêtes et pieds de page
%                                        changement style titres
% 16-fev-2017   G. ANSEL        0.6      mise à jour : modifications pour version 1.6 package imta
%                                        ajout style de page IMTAfancy pour entêtes et pieds de page
%                                        remplacement \listoffigures et \listoftables par \IMTAlistefigures
%                                        et \IMTAlistetableaux
% 17-fev-2017   G. ANSEL        0.7      mise à jour : déplacement définition style IMTAfancy dans imta.sty
%                                        définition commande \IMTAfooter pour personnaliser le pied de page
%                                        en bas à gauche
% 21-fev-2017   T. LE GALL (SC) 0.8      mise à jour : ajout d'un exemple de n° de contrat de recherche
%                                        (pied de page) et vérification de la compilation du document 
% 28-fev-2017   G. ANSEL        0.9      mise à jour : ajout exemple d'utilisation de l'option copyright
%                                        du package imta (informations copyright sur quatrième de couverture)
% 06-fev-2017   T. LE GALL (SC) 1.0      mise à jour : ajout de définitions d'ensembles mathématiques et exemple
%                                        d'édition de bloc dans le texte.  Ajout de la mention FIN DE FICHIER. 
%                                        Test de compilation avec imta.sty (v2.3) = PASS (0 errors, 0 warnings).
% 06-avr-2017   T. LE GALL (SC) 1.1      mise à jour : ajout du type de document et du numéro de rapport
%                                        de recherche en tant que paramètre du document et affichage des
%                                        informations en première de couverture (suite à demande de la DRI)
% 03-aoû-2017   T. LE GALL (SC) 1.2      mise à jour : exemple de section de texte sur deux colonnes et équations
%                                        dans deux blocs adjacents de style Beamer
% 09-aoû-2017   T. LE GALL (SC) 1.3      mise à jour : ajout des équations (matrices, vecteurs)
% 11-sep-2017   T. LE GALL (SC) 1.4      mise à jour : attribution du n°ISSN 2556-5060 pour les rapports de recherche
% 22-sep-2017   G. ANSEL        1.5      mise à jour : ajout d'un exemple d'insertion de tableau
% 25-sep-2017   T. LE GALL      1.6      mise à jour : légende du tableau et n° de version du modèle en page de garde
%                                        + tests de non-négression du modèle (env. Win 7) suite à BUGFIX_21092017_1
%                                         et BUGFIX_21092017_2 dans le fichier imta.sty v2.8 et v2.9
% 27-nov-2018   G. ANSEL        1.7      mise à jour : modifications pour passage en version 3.5 du package imta
% 21-jan-2019   T. LE GALL (SC) 1.8      mise à jour : couverture : type de document au-dessus de titre et liste
%                                        des auteurs et contributeurs sous le titre + tests de compilation (Win10)
%                                        + tests de compilation (UBUNTU 18.04 LTS)
% 10-mai-2020  T. LE GALL (SC)  1.9      mise à jour : ajout d'exemples supplémentaires (équations)
%
% 25-jan-2021  T. LE GALL (MEE) 2.0      mise à jour : département Mathematical & Electrical Engineering
%
% 18-oct-2023  T. LE GALL (MEE) 2.2      mise à jour : 1ere + 4eme de courverture suite à demandes DRI (N. Fontaine)
%                                        - 1ere de couvertture: nouveau format et procédure de n°de rapport de recherche
%                                        - 1ere de couvertture: mentions des niveaux de classification suivant ZRR
%                                        - 4ème de couverture:
%                                          suppression site de Toulouse + maj.carte
%                                          suppression ISSN remplacé par CreativeComms
%                                          ajouts des icones cliquables (réseaux sociaux)                           
%
%
% NOTES/COMMENTAIRES :
%
% - Exemple de rapport scientifique au format IMT Atlantique sous LaTex avec insertion de formules,
%   figures, ref. biblio, hyperliens, signets, renvois actifs, surlignage des formules. Le fichier
%   d'édition tp_latex_006.tex utilise le fichier de style imta.sty. Cf. fichier lab_006_readme.txt
%   situé sous le répertoire lab_006.
%
% REFERENCES :
%
% - cf. bibliographie
% 
%***********************************************************************************************

%-----------------------------------------------------------------------------------------------
%                                    DEBUT DU FICHIER D'EDITION
%-----------------------------------------------------------------------------------------------

\documentclass[11pt,a4paper,french]{article}

\usepackage{babel}
\usepackage[utf8]{inputenc}
\usepackage[T1]{fontenc}

\usepackage{newtxtext} % Police pour texte
\usepackage{newtxmath} % Police pour formules
\usepackage{textcomp}
\usepackage{graphicx}
\usepackage{amsmath}
\usepackage{microtype}

\usepackage{framed}	% Permet de surligner les résultats importants

% L'option "copyright" affiche la licence Creative Common en 4eme de couverture
% Les informations de copyright sont paramétrables, ci-dessous :

\usepackage[copyright]{../sty/imta} % édition avec copyright en 4eme de couverture
%\usepackage{../sty/imta}	% édition sans copyright en 4eme de couverture

% Signets et hyperliens (à charger en dernier)
\usepackage[colorlinks=true,linkcolor=black,urlcolor=blue,citecolor=IMTAbleu]{hyperref}
% Mise en évidence de textes par icônes et barre verticale (à charger en dernier)
\usepackage{../sty/awesomebox}

%----------------- Métadonnées du PDF (facilite le travail des moteurs de recherche) -----------
\hypersetup{
	pdfinfo = {
		Author = {<Auteur 1>; <Auteur 2>; ...},
		Title = {<Titre du document>},
		Keywords = {<Mot-clé 1>; <Mot-clé 2>; ...}
	}
}

%--------------------------- Définition du contenu du pied de page ----------------------------

% Le pied de page contient par défaut le titre du document.
% Il peut être remplacé en décommentant l'une des lignes ci-dessous

%\IMTAfooter{Modèle de document \LaTeX{} IMT Atlantique} % Exemple de remplacement par un autre texte
\IMTAfooter{\thereportnumber} % Utilisation du numéro de rapport de recherche

%-------------------- Informations de copyright pour quatrième de couverture ------------------
%          (uniquement si l'option "copyright" est présente pour le package imta)

\IMTCopyright{[Choisir une licence Creative Commons\\\URLcclicence\\
	           et l’apposer ici à la place de cette phrase en enlevant les crochets]}

%----------------- Définitions d'ensembles et d'opérateurs mathématiques ----------------------

\def\Cset{\mathbb{C}} % complexes
\def\Hset{\mathbb{H}} % hilbert
\def\Nset{\mathbb{N}} % entiers naturels
\def\Qset{\mathbb{Q}} % rationnels
\def\Rset{\mathbb{R}} % reels
\def\Zset{\mathbb{Z}} % entiers relatifs
\def\Dset{\mathbb{D}} % disque unite ouvert
\def\Eset{\mathbb{E}} % esperance mathematique

%-------------------------------------- FIN DES DEFINITIONS ----------------------------------

\begin{document}

%---------------------------------------------------------------------------------------------
%                                   PAGE DE TITRE (please modify)
%---------------------------------------------------------------------------------------------

\entity{\textbf{\large{IMT Atlantique}}\\
	%Dépt. Automatique productique et informatique\\
	%Dépt. Informatique\\ 
	%Dépt. Image \& traitement de l'information\\
	%Dépt. Langues \& culture internationale\\ 
	%Dépt. Logique des usages, sciences sociales \& sciences de l'information\\ 
	%Dépt. Micro-ondes\\ 
	%Dépt. Optique\\
	%Dépt. Physiques subatomiques et technologies\\
	%Dépt. Sciences sociales et de gestion\\
	% Dépt. Mathematical \& Electrical Engineering\\
	%Dépt. Systèmes énergétiques et environnement\\
	%Dépt. Systèmes réseaux, cybersécurité et droit du numérique\\
	Technopôle de Brest-Iroise - CS 83818\\
	29238 Brest Cedex 3\\
	Téléphone: +33 (0)2 29 00 13 04\\
	Télécopie: +33 (0)2 29 00 10 12\\
	%4, rue Alfred Kastler\\
	%CS 20722\\
	%44307 Nantes Cedex 3\\
	%Téléphone: +33 (0)2 51 85 81 00\\
	%Télécopie: +33 (0)2 99 12 70 08\\		
	%2, rue de la Châtaigneraie\\
	%CS 17607\\
	%35576 Cesson Sévigné Cedex\\
	%Téléphone: +33 (0)2 99 12 70 00\\
	%Télécopie: +33 (0)2 51 85 81 99\\
URL: \URLimta
} % end of \entity

% Type de document
%\DocumentType{<Type de document>}  
\DocumentType{Rapport}
%\DocumentType{Support de cours}
%\DocumentType{Support de TP}
%\DocumentType{Mémoire technique}

% Numéro de rapport de recherche (à remplacer)
% Commenter la ligne ci-dessous si le document n'est pas un rapport de recherche (sinon modifier)
\ReportNumber{FIP 1A - SIT 151}

% Commenter la mention de support ci-dessous une fois obtenu le n° du rapport de recherche
% \MentionSupport{Pour obtenir le numéro de rapport envoyer un mail à:\\\EMAILDRI\\
		        % (et ne pas oublier de supprimer cette phrase !)}
		
% Commenter/Décommenter la mention spéciale suivant la sensibilité d'information (obligatoire si ZRR)
% d'après la nomenclature du rapport d'audit CapGemini 2022 "Rapport d'écart - v1.1 p17: 
% PUBLIC       : impact null si divulgation hors-liste de diffusion
% LIMITE       : impact modéré si divulgation hors-liste de diffusion
% CONFIDENTIEL : impact important si divulgation hors-liste de diffusion
% RESTREINT    : impact catastrophique si divulgation hors-liste de diffusion

% \MentionSpeciale{DIFFUSION PUBLIQUE}
%\MentionSpeciale{\textcolor{red}{\fbox{DIFFUSION LIMITÉE}}}
%\MentionSpeciale{\textcolor{red}{\fbox{DIFFUSION CONFIDENTIELLE}}}
\MentionSpeciale{\textcolor{red}{\fbox{DIFFUSION RESTREINTE}}}   
             

\title{articulation machine-langage-os}
% \title{Étude d'une forme d'onde pour communications longues-distances par le canal acoustique sous-marin}

\author{
	Martin Baumgaertner
}

% \contributors{Contributeur 1, Contributeur 2...}

% Numéro de version du document, affiché sur la couverture
% Commenter cette ligne pour ne pas faire apparaître la version du document.
\version{1.0}

% Date d'édition du document
\date{\today}

%------------------------------------------ Do not modify ------------------------------------
\IMTAfrontcover
\pagestyle{IMTAfancy} % Changement du style de page pour avoir en-tête et pied de page
%--------------------------------------- End of do not modify --------------------------------

%% Affichage des listes et tables.
\IMTAsommaire
\newpage
\IMTAlistefigures  % commenter pour ne pas avoir la liste des figures
\IMTAlistetableaux   % commenter pour ne pas avoir la liste des tables
\newpage
%---------------------------------------------------------------------------------------------
%%                                   INTRODUCTION (please modify)
%%---------------------------------------------------------------------------------------------
%
%\section{Introduction} % exemple de structure d'introduction d'après [Lichtfouse 2012]



%---------------------------------------------------------------------------------------------
%                                   CORPS DU DOCUMENT (please modify)
%---------------------------------------------------------------------------------------------

\section{Introduction}\label{sec:S_STRU}
\subsection{Partie 1}


% %--------------------------------------- DEBUT DES ANNEXES --------------------------------------
% \newpage
% \begin{IMTAannexes}
	
% %-------------------------------------------- ANNEXE 1  -----------------------------------------
	
% \IMTAannexe{Exemple d'annexe}\label{sec:S_ANN_EX}

% Un exemple d'annexe.
% \section{Première partie}


% \subsection{Sous-section}
% \subsection{Sous-section}

% \section{Deuxième partie}

% \subsection{Sous-section}
% \subsection{Sous-section}

% %-------------------------------------------- ANNEXE 2  -----------------------------------------

% \IMTAannexe{Autre annexe}\label{sec:S_ANN_EX2}

% Un autre exemple d'annexe.
% \section{Première partie}


% \subsection{Sous-section}
% \subsection{Sous-section}

% \section{Deuxième partie}

% \subsection{Sous-section}
% \subsection{Sous-section}

% %------------------------------------- -- FIN DES ANNEXES --------------------------------------
% \end{IMTAannexes}

% %-----------------------------------DEBUT DE LA BIBLIOGRAPHIE ----------------------------------
% \newpage
% %------------------ Ajout du renvoi vers la bibliographie dans le sommaire ---------------------
% \phantomsection
% \addcontentsline{toc}{section}{\refname}
% %-----------------------------------------------------------------------------------------------

% \begin{thebibliography}{99}
% \bibitem{[Lichtfouse2012]} Eric Lichtfouse, \emph{Rédiger pour être publié}, Springer, $2^{eme}$ édition, 2012
%  (cote bibliothèque IMT Atlantique Brest : $0.343$ LICH).
 
% \bibitem{[URL_LATEX1]} \href{https://www.latex-fr.net/3_composition/texte/paragraphes/encadrer_du_texte}{LaTeX FAQ - Encadrer ou mettre en valeur du texte}
 
% %----------------------------------- FIN DE LA BIBLIOGRAPHIE -----------------------------------
% \end{thebibliography}
%---------------------------------------- Do not modify ----------------------------------------
\IMTAcoverpage
%------------------------------------ End of do not modify -------------------------------------

\end{document}

%***********************************************************************************************
%                                   FIN DU FICHIER D'EDITION
%***********************************************************************************************
